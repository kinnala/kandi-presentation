\documentclass{beamer}
\usepackage[finnish]{babel}
\usepackage[T1]{fontenc}
\usepackage[utf8]{inputenc}
\usepackage{tikz}
\usetheme{Warsaw}
\title[Data assimilation in an elastic friction model]{Data assimilation in an elastic friction model}
\author{Tom Gustafsson}
\date{20. September 2012}
\begin{document}

\begin{frame}
\titlepage
\end{frame}

\begin{frame}{Question}

\begin{itemize}
\item Is it possible to estimate weakly known parameters from a simple elastic friction model by using the tools of data assimilation?
\end{itemize}

\begin{figure}
\includegraphics[width=8cm]{fretting_mesh.pdf}
\end{figure}

\end{frame}

\begin{frame}{Model, 2D}

\begin{itemize}
\item Initial setting: Block, slab
\item Boundary conditions
\end{itemize}

\begin{figure}
\includegraphics[width=10cm]{fretting_geom.pdf}
\end{figure}

\begin{itemize}
\item Abaqus/Standard 6.12-1, simulations on CSC
\end{itemize}

\end{frame}

\begin{frame}{Steps of the simulation}

\begin{itemize}
\item Step 1: 5 kN force acting downwards
\end{itemize}

\begin{figure}
\includegraphics[width=9cm]{mises.pdf}
\end{figure}

\begin{itemize}
\item Step 2: Displacement of block's upper boundary by 70 cm to the right
\end{itemize}

\end{frame}

\begin{frame}{Steps of the simulation: Displacement of the block}

\begin{itemize}
\item Done by using boundary conditions. Thus, a "slow displacement"
\end{itemize}

\begin{figure}
\includegraphics[width=10cm]{anim1.pdf}\\
\includegraphics[width=10cm]{anim2.pdf}
\end{figure}

\end{frame}

\begin{frame}{Steps of the simulation: Displacement of the block, 2}

\begin{figure}
\includegraphics[width=10cm]{anim3.pdf}\\
\includegraphics[width=10cm]{anim4.pdf}\\
\includegraphics[width=10cm]{anim5.pdf}
\end{figure}

\end{frame}

\begin{frame}{Inversion problem}

\begin{itemize}
\item Attempting to estimate friction coefficient $\mu$
\item As \emph{a priori} data: $x$ directional stress in chosen measurement points\\($\sim$ strain gauge)
\end{itemize}

\begin{figure}
\includegraphics[width=10cm]{fretting_geom_meas.pdf}
\end{figure}

%\begin{itemize}
%\item Mittadata synteettistä
%\end{itemize}

\end{frame}

%\begin{frame}{Synteettisen mittadatan generointi}

%\begin{itemize}
%\item Minimoidaan inversiorikosta $\rightarrow$ mittadata tiheämmästä verkosta
%\end{itemize}

%\begin{figure}
%\includegraphics[width=10cm]{finer_mesh.pdf}
%\end{figure}

%\begin{itemize}
%\item Miten verrata tiheämmän ja harvemman verkon antamia jännityksiä?
%\end{itemize}

%\end{frame}

%\begin{frame}{Synteettisen mittadatan generointi 2}

%\begin{figure}
%\includegraphics[width=10cm]{fretting_meas_tarkempi.pdf}
%\end{figure}

%\end{frame}

%\begin{frame}{Ongelman yhteenveto}

%\begin{itemize}
%\item Estimoitava suure: Kitkakerroin $\mu=0{,}5$
%\item Etukäteistieto: $x$-suuntaiset jännitykset mittapisteissä
%\item \emph{Data-assimilaatio}
%\end{itemize}

%\end{frame}

%\begin{frame}{Ennen kuin jatketaan}

%\begin{figure}
%\includegraphics[width=5cm]{2dnormal.pdf}
%$~~$
%\includegraphics[width=5cm]{2dcontour.pdf}
%\end{figure}

%\end{frame}

%\begin{frame}{Ennen kuin jatketaan 2}

%\begin{itemize}
%\item Useampiulotteisen normaalijakauman karakterisoi kovarianssimatriisi $\boldsymbol{\Sigma}$
%\item $\mathcal{N}_k(\boldsymbol{\mu}_0,\boldsymbol{\Sigma})$, jossa $\boldsymbol{\Sigma} \in \mathbb{R}^{k \times k}$
%\item Neliömatriisi, diagonaalilla varianssit eri dimensioissa
%\item Muut alkiot kertovat dimensioiden välisen kovarianssin
%\end{itemize}

%\end{frame}

\begin{frame}{Data assimilation}

\begin{itemize}
\item Gives an answer to the problem: How to merge measurement and model data in an optimal way?
\item Traditional applications: Weather prediction, oceanography
\item Multiple different methods: 3D- and 4DVar, the family of Kalman filters, ...
\item In this work \emph{Ensemble Kalman Filter}
\end{itemize}

\end{frame}

%\begin{frame}{Data-assimilaatio, yleistä}

%\begin{itemize}
%\item Systeemin (todellinen) tila $\boldsymbol{\psi}^t \in \mathbb{R}^N$
%\item Mittaus $\boldsymbol{d} \in \mathbb{R}^M$
%\begin{itemize}
%\item Ei tarkka
%\item Suhde tilaan $\boldsymbol{d} = \mathbf{M}\boldsymbol{\psi}^t+\boldsymbol{\epsilon}$
%\item Mittamatriisi $\mathbf{M} \in \mathbb{R}^{M \times N}$
%\item Virhe $\boldsymbol{\epsilon} \sim \mathcal{N}_M(0,\boldsymbol{\Sigma})$
%\item Kovarianssimatriisi $\boldsymbol{\Sigma} \in \mathbb{R}^{M \times M}$
%\end{itemize}
%\item Ennustettu tila $\boldsymbol{\psi}^f \in \mathbb{R}^N$
%\begin{itemize}
%\item Aluksi esim. mittauksien perusteella
%\end{itemize}
%\end{itemize}

\begin{frame}{Ensemble Kalman Filter}

\begin{itemize}
\item System is characterized by a bunch of state vectors $\boldsymbol{\psi} \in \mathbb{R}^n$
\item They estimate the same \emph{true} state and their deviation characterizes the uncertainty
\item This group of states is known as \emph{ensemble} 
\item Each member of the ensemble is integrated forwards in time to the next measurement point
\item After the time integration each state is merged with the measurement using
\[
\boldsymbol{\psi}^a = \boldsymbol{\psi}^f + \boldsymbol{\Sigma}_\psi \mathbf{M}^\mathrm{T} \left(\boldsymbol{\Sigma}_d+\mathbf{M}\boldsymbol{\Sigma}_\psi\mathbf{M}^\mathrm{T}\right)^{-1}\left(\boldsymbol{d}-\mathbf{M}\boldsymbol{\psi}^f\right)
\]
\end{itemize}

\end{frame}

\begin{frame}{Ensemble Kalman Filter, estimating parameters}

\begin{itemize}
\item Using the same procedure as previously but extend each state vector with the unknown parameters
\item Formulation of the method guarantees that the unknown parameters
\end{itemize}

\end{frame}

\begin{frame}{Ensemble Kalman Filter, yhteenveto}

\begin{figure}
\includegraphics[width=5.5cm]{enkf_luuppi.pdf}
\end{figure}

\end{frame}

\begin{frame}{Takaisin ongelmaan}

\begin{itemize}
\item Malli
\begin{figure}
\includegraphics[width=3.3cm]{anim1.pdf}
\includegraphics[width=3.3cm]{anim3.pdf}
\includegraphics[width=3.3cm]{anim5.pdf}
\end{figure}

\item Estimoitava parametri $\mu$
\item Määritellään tilaksi
\[
  \boldsymbol{\psi} = (\sigma_x^1, \sigma_x^2, \sigma_x^3, \dots, \sigma_x^N, \mu)^\mathrm{T}
\]
\item Alussa ei kosketusta $\Rightarrow$ jännitykset nollia
\item Alkutilan määrää ainoastaan siis $\mu_0$
\end{itemize}

\end{frame}

\begin{frame}{Takaisin ongelmaan 2}

\begin{itemize}
\item Tarvitaan
\begin{itemize}
\item Alkuarvaus $\mu_0=0{,}6$
\item Alkuarvauksen virhe $\sigma_0 = 0{,}1$
\item Kokoelman koko $n=200$
\item Alkukokoelma jakaumasta $\mathcal{N}(\mu_0,\sigma_0^2)$
\end{itemize}
\item Alkukokoelman yksittäinen tila on siis muotoa
\[
\boldsymbol{\psi}_j = (\underbrace{0, 0, \dots, 0, 0,}_{N~\text{kpl}} \mu_0 + \epsilon )^\mathrm{T},~j=1,\dots,n
\]
\item Mitta"hetket": Yläreunan siirtymät $\Delta x = 7,14,21,\dots,70$
\item Mittadata synteettisesti
\end{itemize}

%\begin{itemize}
%\item Alussa ei jännityksiä, joten alkutilan määrää ainoastaan $\mu_0 = 0{,}6$
%\item Oletetaan jokin virhe; tässä $\sigma_0 = 0{,}1$ 
%\item Päätetään kokoelman koko, esim. $n=200$
%\item Generoidaan alkukokoelma jakaumasta $\mathcal{N}(\mu_0,\sigma_0^2)$
%\item Alkukokoelman yksittäinen tila muotoa
%\[
%\boldsymbol{\psi}_j = (\underbrace{0, 0, \dots, 0, 0,}_{N~\text{kpl}} \mu_0 + \epsilon )^\mathrm{T},~j=1,\dots,n
%\]
%\item Mitta"hetket": Yläreunan siirtymä $\Delta x = 7,14,21,\dots,70$
%\item Simuloidaan seuraavaan mittahetkeen asti ja yhteensulautetaan mittaus ja kokoelma $\Rightarrow$ kitkakertoimen estimaatti
%\item Mittahetkiä 10, estimaatteja 10
%\end{itemize}

\end{frame}

\begin{frame}{Synteettisen mittadatan generointi}

\begin{itemize}
\item Minimoidaan inversiorikosta $\rightarrow$ mittadata tiheämmästä verkosta
\end{itemize}

\begin{figure}
\includegraphics[width=10cm]{finer_mesh.pdf}
\end{figure}

\begin{itemize}
\item Miten verrata tiheämmän ja harvemman verkon antamia jännityksiä?
\end{itemize}

\end{frame}

\begin{frame}{Synteettisen mittadatan generointi 2}

\begin{figure}
\includegraphics[width=10cm]{fretting_meas_tarkempi.pdf}
\end{figure}

\end{frame}

\begin{frame}{Tuloksia, $\mathbf{Q}=\mathbf{0}$}


\begin{figure}
\begin{tikzpicture}
\node[above right] (img) at (0,0) {\includegraphics[width=7cm]{xxx_20_no_model_error.pdf}};
\node at (230pt,55pt) {$n=20$};
\end{tikzpicture}
\end{figure}
\vspace{-0.5cm}
\begin{figure}
\begin{tikzpicture}
\node[above right] (img) at (0,0) {\includegraphics[width=7cm]{xxx_200_no_model_error.pdf}};
\node at (230pt,55pt) {$n=200$};
\end{tikzpicture}
\end{figure} 

\end{frame}

\begin{frame}{Tuloksia, mallivirheen vaikutus, $n=200$}

\begin{figure}
\begin{tikzpicture}
\node[above right] (img) at (0,0) {\includegraphics[width=7cm]{xxx_200_no_model_error.pdf}};
\node at (230pt,55pt) {$\mathbf{Q}=\mathbf{0}$};
\end{tikzpicture}
\end{figure}
\vspace{-0.5cm}
\begin{figure}
\begin{tikzpicture}
\node[above right] (img) at (0,0) {\includegraphics[width=7cm]{xxx_200_model_error.pdf}};
\node at (230pt,55pt) {$\mathbf{Q}=\mathbf{I}\sigma$};
\end{tikzpicture}
\end{figure} 

%\begin{figure}
%\includegraphics[width=7cm]{xxx_200_model_error.pdf}
%\end{figure}

%\begin{figure}
%\includegraphics[width=7cm]{xxx_1000_model_error.pdf}
%\end{figure}

\end{frame}

\begin{frame}{Kokoelman koon vaikutus, keskiarvo}

Punainen: $\Delta x=70$, sininen: $\Delta x=42$, vihreä: $\Delta x=14$

\begin{figure}
\includegraphics[width=9cm]{mean_conv_2_6_10.pdf}
\end{figure}

\end{frame}

\begin{frame}{Kokoelman hajonta}

\begin{figure}
\includegraphics[width=9cm]{ensemble_stdev.pdf}
\end{figure}

\end{frame}

\begin{frame}{Peräkkäisten analyysien varianssi}

\begin{figure}
\includegraphics[width=9cm]{stdev_between_analysis.pdf}
\end{figure}

\end{frame}

\begin{frame}{Kysymyksiä?}


\end{frame}

\end{document}
